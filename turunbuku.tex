% Options for packages loaded elsewhere
\PassOptionsToPackage{unicode}{hyperref}
\PassOptionsToPackage{hyphens}{url}
%
\documentclass[
]{book}
\usepackage{lmodern}
\usepackage{amssymb,amsmath}
\usepackage{ifxetex,ifluatex}
\ifnum 0\ifxetex 1\fi\ifluatex 1\fi=0 % if pdftex
  \usepackage[T1]{fontenc}
  \usepackage[utf8]{inputenc}
  \usepackage{textcomp} % provide euro and other symbols
\else % if luatex or xetex
  \usepackage{unicode-math}
  \defaultfontfeatures{Scale=MatchLowercase}
  \defaultfontfeatures[\rmfamily]{Ligatures=TeX,Scale=1}
\fi
% Use upquote if available, for straight quotes in verbatim environments
\IfFileExists{upquote.sty}{\usepackage{upquote}}{}
\IfFileExists{microtype.sty}{% use microtype if available
  \usepackage[]{microtype}
  \UseMicrotypeSet[protrusion]{basicmath} % disable protrusion for tt fonts
}{}
\makeatletter
\@ifundefined{KOMAClassName}{% if non-KOMA class
  \IfFileExists{parskip.sty}{%
    \usepackage{parskip}
  }{% else
    \setlength{\parindent}{0pt}
    \setlength{\parskip}{6pt plus 2pt minus 1pt}}
}{% if KOMA class
  \KOMAoptions{parskip=half}}
\makeatother
\usepackage{xcolor}
\IfFileExists{xurl.sty}{\usepackage{xurl}}{} % add URL line breaks if available
\IfFileExists{bookmark.sty}{\usepackage{bookmark}}{\usepackage{hyperref}}
\hypersetup{
  pdftitle={Untirta dalam Angka: semester genap Tahun Akademik 2019/2020},
  pdfauthor={oleh Subbagian Registrasi dan Statistik},
  hidelinks,
  pdfcreator={LaTeX via pandoc}}
\urlstyle{same} % disable monospaced font for URLs
\usepackage{longtable,booktabs}
% Correct order of tables after \paragraph or \subparagraph
\usepackage{etoolbox}
\makeatletter
\patchcmd\longtable{\par}{\if@noskipsec\mbox{}\fi\par}{}{}
\makeatother
% Allow footnotes in longtable head/foot
\IfFileExists{footnotehyper.sty}{\usepackage{footnotehyper}}{\usepackage{footnote}}
\makesavenoteenv{longtable}
\usepackage{graphicx,grffile}
\makeatletter
\def\maxwidth{\ifdim\Gin@nat@width>\linewidth\linewidth\else\Gin@nat@width\fi}
\def\maxheight{\ifdim\Gin@nat@height>\textheight\textheight\else\Gin@nat@height\fi}
\makeatother
% Scale images if necessary, so that they will not overflow the page
% margins by default, and it is still possible to overwrite the defaults
% using explicit options in \includegraphics[width, height, ...]{}
\setkeys{Gin}{width=\maxwidth,height=\maxheight,keepaspectratio}
% Set default figure placement to htbp
\makeatletter
\def\fps@figure{htbp}
\makeatother
\setlength{\emergencystretch}{3em} % prevent overfull lines
\providecommand{\tightlist}{%
  \setlength{\itemsep}{0pt}\setlength{\parskip}{0pt}}
\setcounter{secnumdepth}{5}
\usepackage{booktabs}
\usepackage{booktabs}
\usepackage{longtable}
\usepackage{array}
\usepackage{multirow}
\usepackage{wrapfig}
\usepackage{float}
\usepackage{colortbl}
\usepackage{pdflscape}
\usepackage{tabu}
\usepackage{threeparttable}
\usepackage{threeparttablex}
\usepackage[normalem]{ulem}
\usepackage{makecell}
\usepackage{xcolor}
\usepackage[]{natbib}
\bibliographystyle{apalike}

\title{Untirta dalam Angka: semester genap Tahun Akademik 2019/2020}
\author{oleh Subbagian Registrasi dan Statistik}
\date{2020-03-12}

\begin{document}
\maketitle

{
\setcounter{tocdepth}{1}
\tableofcontents
}
\hypertarget{mahasiswa-baru}{%
\chapter{Mahasiswa Baru}\label{mahasiswa-baru}}

Placeholder

\hypertarget{mahasiswa-aktif}{%
\chapter{Mahasiswa Aktif}\label{mahasiswa-aktif}}

\begin{quote}
Mahasiswa aktif di semester genap Tahun Akademik 2019/2020
\end{quote}

\begin{table}[H]
\centering
\begin{tabular}{l|r|r|r|r|r|r|r|r}
\hline
\multicolumn{1}{c|}{ } & \multicolumn{7}{c|}{Angkatan} & \multicolumn{1}{c}{ } \\
\cline{2-8}
Fakultas & 2013 & 2014 & 2015 & 2016 & 2017 & 2018 & 2019 & Jumlah\\
\hline
\multicolumn{9}{l}{\textbf{Pascasarjana}}\\
\hline
\hspace{1em}HUKUM (S2) & 0 & 0 & 0 & 23 & 20 & 40 & 78 & 161\\
\hline
\hspace{1em}ILMU PERTANIAN & 0 & 0 & 0 & 0 & 9 & 17 & 21 & 47\\
\hline
\hspace{1em}MAGISTER ADMINISTRASI PUBLIK & 0 & 0 & 0 & 16 & 3 & 33 & 37 & 89\\
\hline
\hspace{1em}MAGISTER AKUNTANSI & 0 & 0 & 0 & 14 & 2 & 20 & 29 & 65\\
\hline
\hspace{1em}MAGISTER MANAJEMEN & 0 & 0 & 0 & 14 & 7 & 65 & 69 & 155\\
\hline
\hspace{1em}PENDIDIKAN BAHASA INDONESIA (S2) & 0 & 0 & 0 & 2 & 0 & 19 & 9 & 30\\
\hline
\hspace{1em}PENDIDIKAN BAHASA INGGRIS & 0 & 0 & 0 & 17 & 6 & 28 & 10 & 61\\
\hline
\hspace{1em}PENDIDIKAN MATEMATIKA S2 & 0 & 0 & 0 & 9 & 5 & 22 & 14 & 50\\
\hline
\hspace{1em}TEKNOLOGI PENDIDIKAN (S2) & 0 & 0 & 0 & 4 & 8 & 53 & 58 & 123\\
\hline
\hspace{1em}TEKNIK KIMIA (S2) & 0 & 0 & 0 & 0 & 0 & 0 & 35 & 35\\
\hline
\hspace{1em}ILMU KOMUNIKASI (S2) & 0 & 0 & 0 & 0 & 0 & 0 & 16 & 16\\
\hline
\multicolumn{9}{l}{\textbf{Hukum}}\\
\hline
\hspace{1em}HUKUM (S1) & 43 & 49 & 158 & 364 & 369 & 388 & 365 & 1736\\
\hline
\multicolumn{9}{l}{\textbf{FKIP}}\\
\hline
\hspace{1em}PENDIDIKAN BAHASA INDONESIA (S1) & 9 & 5 & 19 & 66 & 80 & 106 & 95 & 380\\
\hline
\hspace{1em}PENDIDIKAN BAHASA INGGRIS & 7 & 12 & 19 & 112 & 122 & 131 & 110 & 513\\
\hline
\hspace{1em}PENDIDIKAN GURU PENDIDIKAN ANAK USIA DINI & 9 & 7 & 19 & 72 & 77 & 62 & 48 & 294\\
\hline
\hspace{1em}PENDIDIKAN PANCASILA DAN KEWARGANEGARAAN & 0 & 4 & 20 & 46 & 48 & 34 & 69 & 221\\
\hline
\hspace{1em}PENDIDIKAN VOKASIONAL TEKNIK ELEKTRO & 0 & 18 & 11 & 23 & 32 & 28 & 54 & 166\\
\hline
\hspace{1em}PENDIDIKAN VOKASIONAL TEKNIK MESIN & 0 & 11 & 19 & 20 & 22 & 20 & 42 & 134\\
\hline
\hspace{1em}PENDIDIKAN MATEMATIKA & 8 & 4 & 30 & 94 & 111 & 99 & 89 & 435\\
\hline
\hspace{1em}PENDIDIKAN FISIKA & 0 & 0 & 3 & 35 & 38 & 32 & 57 & 165\\
\hline
\hspace{1em}PENDIDIKAN KIMIA & 0 & 2 & 20 & 42 & 31 & 35 & 64 & 194\\
\hline
\hspace{1em}PENDIDIKAN BIOLOGI & 10 & 32 & 75 & 103 & 93 & 102 & 94 & 509\\
\hline
\hspace{1em}PENDIDIKAN SEJARAH & 0 & 4 & 25 & 37 & 43 & 34 & 66 & 209\\
\hline
\hspace{1em}PENDIDIKAN NON FORMAL & 3 & 9 & 24 & 60 & 61 & 103 & 85 & 345\\
\hline
\hspace{1em}PENDIDIKAN KHUSUS & 0 & 0 & 9 & 41 & 47 & 41 & 62 & 200\\
\hline
\hspace{1em}PENDIDIKAN GURU SEKOLAH DASAR & 7 & 4 & 24 & 126 & 129 & 126 & 113 & 529\\
\hline
\hspace{1em}BIMBINGAN DAN KONSELING & 0 & 1 & 15 & 54 & 46 & 51 & 77 & 244\\
\hline
\hspace{1em}PENDIDIKAN SENI PERTUNJUKAN & 0 & 0 & 27 & 41 & 39 & 30 & 57 & 194\\
\hline
\hspace{1em}PENDIDIKAN SOSIOLOGI & 0 & 3 & 19 & 52 & 39 & 40 & 72 & 225\\
\hline
\hspace{1em}PENDIDIKAN IPA & 0 & 2 & 9 & 32 & 39 & 36 & 71 & 189\\
\hline
\multicolumn{9}{l}{\textbf{Teknik}}\\
\hline
\hspace{1em}TEKNIK INDUSTRI & 6 & 7 & 16 & 106 & 90 & 78 & 101 & 404\\
\hline
\hspace{1em}TEKNIK MESIN & 18 & 12 & 30 & 72 & 74 & 67 & 86 & 359\\
\hline
\hspace{1em}TEKNIK SIPIL & 11 & 9 & 24 & 104 & 95 & 69 & 75 & 387\\
\hline
\hspace{1em}TEKNIK ELEKTRO & 26 & 34 & 47 & 81 & 93 & 61 & 92 & 434\\
\hline
\hspace{1em}TEKNIK KIMIA & 8 & 22 & 47 & 83 & 97 & 85 & 106 & 448\\
\hline
\hspace{1em}TEKNIK METALURGI & 9 & 20 & 36 & 75 & 81 & 61 & 92 & 374\\
\hline
\multicolumn{9}{l}{\textbf{Pertanian}}\\
\hline
\hspace{1em}AGRIBISNIS & 9 & 15 & 25 & 115 & 83 & 127 & 158 & 532\\
\hline
\hspace{1em}AGROEKOTEKNOLOGI & 17 & 10 & 24 & 92 & 88 & 115 & 139 & 485\\
\hline
\hspace{1em}ILMU PERIKANAN & 13 & 15 & 35 & 57 & 70 & 64 & 87 & 341\\
\hline
\hspace{1em}TEKNOLOGI PANGAN & 0 & 0 & 0 & 0 & 0 & 42 & 93 & 135\\
\hline
\multicolumn{9}{l}{\textbf{FEB}}\\
\hline
\hspace{1em}AKUNTANSI D3 & 0 & 0 & 5 & 2 & 70 & 67 & 52 & 196\\
\hline
\hspace{1em}AKUNTANSI S1 & 0 & 0 & 0 & 0 & 0 & 2 & 0 & 2\\
\hline
\hspace{1em}ILMU EKONOMI PEMBANGUNAN & 15 & 42 & 55 & 76 & 79 & 69 & 78 & 414\\
\hline
\hspace{1em}EKONOMI SYARIAH & 0 & 0 & 19 & 38 & 48 & 44 & 36 & 185\\
\hline
\hspace{1em}MANAJEMEN & 1 & 7 & 21 & 216 & 127 & 172 & 135 & 679\\
\hline
\hspace{1em}AKUNTANSI & 12 & 24 & 56 & 156 & 138 & 145 & 130 & 661\\
\hline
\hspace{1em}PERBANKAN DAN KEUANGAN & 0 & 0 & 1 & 3 & 38 & 37 & 31 & 110\\
\hline
\hspace{1em}MANAJEMEN PEMASARAN (D3) & 0 & 2 & 0 & 5 & 30 & 21 & 14 & 72\\
\hline
\hspace{1em}PERPAJAKAN & 0 & 0 & 4 & 8 & 40 & 64 & 50 & 166\\
\hline
\multicolumn{9}{l}{\textbf{FISIP}}\\
\hline
\hspace{1em}ADMINISTRASI PUBLIK & 22 & 22 & 52 & 111 & 135 & 151 & 142 & 635\\
\hline
\hspace{1em}ILMU KOMUNIKASI & 15 & 17 & 51 & 132 & 129 & 157 & 184 & 685\\
\hline
\hspace{1em}ILMU PEMERINTAHAN & 0 & 7 & 48 & 101 & 100 & 121 & 102 & 479\\
\hline
\multicolumn{9}{l}{\textbf{Kedokteran}}\\
\hline
\hspace{1em}KEPERAWATAN D3 & 0 & 0 & 0 & 1 & 118 & 113 & 116 & 348\\
\hline
\hspace{1em}KEDOKTERAN & 0 & 0 & 0 & 0 & 0 & 0 & 50 & 50\\
\hline
\hspace{1em}GIZI & 0 & 0 & 0 & 0 & 0 & 0 & 39 & 39\\
\hline
\hspace{1em}ILMU KEOLAHRAGAAN & 0 & 0 & 0 & 0 & 0 & 0 & 31 & 31\\
\hline
Jumlah & 278 & 432 & 1141 & 3153 & 3349 & 3727 & 4285 & 16365\\
\hline
\end{tabular}
\end{table}

\hypertarget{dosen}{%
\chapter{Dosen}\label{dosen}}

Placeholder

\hypertarget{rasio-dosen-dan-mahasiswa}{%
\chapter{Rasio Dosen dan Mahasiswa}\label{rasio-dosen-dan-mahasiswa}}

Under construction

\hypertarget{daya-tampung}{%
\chapter{Daya Tampung}\label{daya-tampung}}

Some \emph{significant} applications are demonstrated in this chapter.

\hypertarget{example-one}{%
\section{Example one}\label{example-one}}

Makanan yang sangat mahal

\hypertarget{example-two}{%
\section{Example two}\label{example-two}}

\hypertarget{final-words}{%
\chapter{Final Words}\label{final-words}}

We have finished a nice book.

\hypertarget{tentang-kami}{%
\chapter*{Tentang Kami}\label{tentang-kami}}
\addcontentsline{toc}{chapter}{Tentang Kami}

Registrasi dan Statistik adalah salah satu subbagian di bawah Bagian Akademik dan Kemahasiswaan di Biro Akademik, Kemahasiswaan, dan Perencanaan (BAKP) Universitas Sultan Ageng Tirtayasa. Subbagian Registrasi dan Statistik dikoordinasi oleh seorang Kepala Subbagian dan dibantu oleh 3 orang staf. Berikut adalah tugas pokok Subbagian Registrasi dan Statistik Universitas Sultan Ageng Tirtayasa.

\begin{itemize}
\tightlist
\item
  Makan
\item
  Minum
\end{itemize}

  \bibliography{book.bib,packages.bib}

\end{document}
